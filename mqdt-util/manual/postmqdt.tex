\documentclass[11pt]{article}

\usepackage{amsmath, amssymb, graphicx, subfigure}
\usepackage[numbers]{natbib}
\usepackage{listings}

\usepackage{graphicx}% Include figure files
\usepackage{dcolumn}% Align table columns on decimal point
\usepackage{bm}% bold math
\usepackage[
left=0.7in,right=0.7in,top=0.8in,bottom=0.9in
]{geometry}
\tolerance=5000
\hyphenpenalty=1000
\usepackage{latexsym}
\usepackage{epsfig}
\usepackage{array}
\usepackage{setspace}
\usepackage{lineno}
\usepackage{xcolor}
\usepackage[colorlinks,linkcolor=blue,anchorcolor=blue,urlcolor  = blue,citecolor=blue]{hyperref}
\usepackage{ulem}
\usepackage{txfonts}
\usepackage[section]{placeins}
\newcommand*\funit{ph/\ensuremath{\mu}m\ensuremath{^2}}
\newcommand*\fluenceone[1]{10\ensuremath{^{#1}}~\funit}
\newcommand*\fluence[2]{#1\ensuremath{\times}10\ensuremath{^{#2}}~\funit}

\newcommand*\iunit{W/cm\ensuremath{^2}}
\newcommand*\intensityone[1]{10\ensuremath{^{#1}}~\iunit}
\newcommand*\intensity[2]{#1\ensuremath{\times}10\ensuremath{^{#2}}~\iunit}

\newcommand*{\addheight}[2][.5ex]{%
	\raisebox{0pt}[\dimexpr\height+(#1)\relax]{#2}%
}

\newlength{\figurewidth}
\setlength{\figurewidth}{0.85\textwidth}

\usepackage{amsmath, amssymb, graphicx, subfigure}
\usepackage[numbers]{natbib}
\textwidth = 6.5 in
\textheight = 9.6 in
\oddsidemargin = 0.0 in
\evensidemargin = 0.0 in
\topmargin = 0.0 in
\headheight = 0.0 in
\headsep = -0.3 in
\parskip = 0.2in
\parindent = 0.0in

\renewcommand{\labelenumi}{(\roman{enumi})}

\newcommand{\ket}[1]{\ensuremath{\vert#1\rangle}}
\newcommand{\bra}[1]{\ensuremath{\langle#1\vert}}
\newcommand{\braket}[2]{\ensuremath{\langle#1\vert#2\rangle}}
\newcommand{\braketoperator}[3]{\ensuremath{\left\langle#1\left\lvert#2\right\rvert#3\right\rangle}}
\newcommand{\reducedmatrix}[3]{\ensuremath{\left\langle#1\left\lVert#2\right\rVert#3\right\rangle}}
\newcommand{\pd}[2]{\ensuremath{\frac{\partial#1}{\partial#2}}}
\newcommand{\degree}{\ensuremath{^\circ}}
\newcommand{\config}[3]{1s$^{#1}$2s$^{#2}$2p$^{#3}$}

\setlength{\figurewidth}{0.9\textwidth}

\definecolor{codegreen}{rgb}{0,0.6,0}
\definecolor{codegray}{rgb}{0.5,0.5,0.5}
\definecolor{codepurple}{rgb}{0.58,0,0.82}
\definecolor{backcolour}{rgb}{0.95,0.95,0.92}

\lstdefinestyle{mystyle}{
	backgroundcolor=\color{backcolour},   
	commentstyle=\color{codegreen},
	keywordstyle=\color{magenta},
	numberstyle=\tiny\color{codegray},
	stringstyle=\color{codepurple},
	basicstyle=\ttfamily\footnotesize,
	breakatwhitespace=false,         
	breaklines=true,                 
	captionpos=b,                    
	keepspaces=true,                 
	numbers=left,                    
	numbersep=5pt,                  
	showspaces=false,                
	showstringspaces=false,
	showtabs=false,                  
	tabsize=2
}

\lstset{style=mystyle}




\begin{document}

\title{Manual for MQDT post process}

\author{Rui Jin%$^{1}$
	\\
	\\
	%$^1$Center for Free-Electron Laser Science, DESY, Hamburg, Germany
}

\normalem
\date{\today}
                     
\maketitle

\section{\label{sec:theory}Theory background}
Following the manual for MQDT solver,%\cite{mqdt_man2021},
the discrete levels can be obtained by searching the cross point of 
\begin{subequations}
	\begin{align}
	\label{eq:mqdt}
	F({x_i},{U_{i\alpha},\mu_\alpha})=\det[U_{i\alpha}\sin\pi(x_i+\mu_\alpha)]=0,
	\end{align}
	\begin{align}\label{eq:energy}
	\nu_y(\nu_x) = \frac{q}{\left[I_y - I_x + q^2/\nu^{2}_x\right]^{1/2}}
	\end{align}
	where,
	\begin{align}
	x_i =
	\begin{cases}
	{\displaystyle \nu_i} & \mathrm{for}\ i\in \mathrm{closed\ channels,}\\
	{\displaystyle -\tau } &  \mathrm{for}\ i\in \mathrm{open\ channels,}
	\end{cases} \notag
	\end{align}
\end{subequations}
For the autoionization region, the center of resonances are at local maximum of the energy derivative of phase shift,
\begin{align}
   \label{eq:deriv}
   \frac{d\tau}{dE} = \frac{v_x^3}{2q^2}\frac{d\tau}{dv_x}. 
\end{align}
the local maximum can be found by searching for root of the derivative of expression~(\ref{eq:deriv}).
\begin{align}
\label{eq:sec-deriv}
\frac{d}{dv_x}\left(\frac{v_x^3}{2q^2}\frac{d\tau}{dv_x}\right) |_{v_{res}} = 0. 
\end{align}

%Oscillator strength (density) 

\section{\label{sec:program}Program structure} 

\begin{lstlisting}[language=python]
 constants     # global constants and parameters
 type_def      # user defined complex data structure for easier data communication
 numerical     # frequently used numerical algorithms
 file_cmdl_io  # file i/o and commandline parser
 stdio         # frequently used string operations and block output.
 envset        # seting up machine-dependent systematic parameters (EPS, PI, Nan ...)
 interpolate   # a set of interpolation functions, employed in root finding algorithms.
 darray        # dynamical array for easier data storage
 search_root   # search levels and resonances and create readable and easy to plot files, if provide energy levels, generate a comparison file. 
\end{lstlisting}

\if 0
\begin{figure}[b]
	\includegraphics[width=\figurewidth]{{program-scheme}.png} 
	\caption{\label{fig:program-schem}Schematic diagram of program structure.}
\end{figure}
\fi
\section{\label{sec:numerical}Numerical algorithms}
\begin{enumerate}
\item discrete level finding
\begin{itemize}
	\item  First roughly search for cross\\
	Scan x axis for $\nu_x^{f}$ where the sign of the difference of $G(\nu_x) = F({x_i},{U_{i\alpha},\mu_\alpha}) - \nu_y(\nu_x)$ flips. 
	\item  Finer search\\
	Choose a suitable vicinity of $\nu_x^{f}$, find the root of the difference function $G(\nu_x)$. Where the actual value of  $F({x_i},{U_{i\alpha},\mu_\alpha})$ and $\nu_y(\nu_x)$ are calculated on-the-fly ($F({x_i},{U_{i\alpha},\mu_\alpha})$ is obtained by spline-interpolation).  
\end{itemize}
\item resonance finding
\begin{itemize}
	\item First rought search\\ 
	The Local peaks are  $\frac{d}{dE}\tau(x_{i-1}) <=\frac{d}{dE}\tau(x_{i})<=\frac{d}{dE}\tau(x_{i+1})$. There might be small fluctuations of the derivative due to numerical operations(especially two-point interpolation is used). So we need to filter out some of them.
	\item Noise filter\\
	    noise filter threshold is automatically determined based on the widest resonance width $\max\{\delta_{\alpha}\}$. The widths are  estimated by the imaginary part of projected $K_{cc}$ matrix. ( J. M. Lecomte, J. Phys. B: At. Mol. Phys. 20 (1987) 3645-3662.)\\
	    Or you can manually specify one by adding option [\emph{-filter XXX}] to the command line. 
	\item Finer search\\
	    Employ bisect searching algorithm to search for the derivative of  $\frac{d}{dE}\tau$ function in a suitable vicinity. The function and it's derivative are Spline-interpolated on-the-fly.
	\item Eigenchannel sort
\end{itemize}
\end{enumerate} 
\section{\label{sec:input}Program input}
Required input files:
\begin{lstlisting}[language=bash]
miuang.out    # S matrix prepared by smooth-toolkits
[dalfa.in]    # dipole matrix moment, if provided, automatically calculate oscillator strength (density)
mqdt.in       # MQDT solver control file 
Anorm.out     # Normalized A-coefficients
Ara.out       # Unnormalized A-coefficients                
Dn.out        # Normalization factor N for A-coefficients.
[NormDos.out] # Density of States if discrete calculation  
[elev.exp]    # experimental levels(resonances) to compare. (If provided, the program will generate comparison files automatically.)                    
\end{lstlisting}

The major control file \emph{mqdt.in} is the \textcolor{red}{same} as that for mqdt solver. You can add \textcolor{red}{blank lines and $\#$ to comment. You can also use $\{\}$ to wrap the data block, you can put them in one line or break the wrapper into multiple lines}. 

\begin{table}[]
\caption{Options in \emph{mqdt.in}}
\label{table:mqdt.in}
\begin{tabular}{|p{5cm}|p{10.5cm}|}
\hline
\verb|Z| & target charge \\
\verb|nchan|& number of physical channels (\textcolor{red}{optional}, program will determine it from S-matrix input file) \\
\verb|nop|& number of open channels (\textcolor{red}{optional}, program will determine it from the energy region and the choice of x-axis threshold \verb|IPx|. )\\
\verb|nip|& number of ionization potentials (IP), (\textcolor{red}{optional}, program will determine from the IP array block \verb|IP = {....} |).\\
\verb|IP_unit|& \textcolor{red}{optional} unit for IP, cm-1, ryd, au, a.u., ev, Hartree are accepted (case-insensitive). It will be treated as \textcolor{red}{cm-1} if not specified\\
\verb|IP|& IP array block \verb|IP = {....}|, the \verb|{}| wrapper accept \verb|,| and space to divide strings. \\
\verb|IP_seq|& indices to assign IP to each physical channel.\\

\verb|IPy|& threshold to present effective quantum defect/phase-shift. channels associated to threshold $IP \leq IP_y$ will be treated \textcolor{red}{opened}. You \textcolor{red}{must} specify it for \textcolor{red}{bound-state} calculation. It will be determined internally in \textcolor{red}{continuum} calculation.\\
\verb|IPx|& threshold to present energy. channels associated to threshold $IP \geq IP_x$ will be treated \textcolor{red}{closed}. You \textcolor{red}{must} specify it for \textcolor{red}{bound-state} calculation. It will be determined internally in \textcolor{red}{continuum} calculation.\\

\verb|xrange|& \textcolor{red}{optional} $\nu_x$ with respect to $IP_x$. You must either specify \verb|xrange| or \verb|E_continuum|.\\
\verb|E_continuum|&\textcolor{red}{optional} continuum electron energy with respect to the first IP $I_1$. You can specify unit for the energy range string e.g. \verb|E_continuum = {0.0:0.20_ryd}|. Accepted energy units are IP, cm-1, ryd, au, a.u., ev, Hartree (case-insensitive). The unit is \textcolor{red}{cm-1} by default.\\
\verb|x_grid_type|& Options: mu-eigen / mu-proj / uniform\\
\verb|y_grid_type|& Options: adaptive / uniform\\
\verb|nx_flat|& number of grid per X-grid segment for the flat region.\\
\verb|nx_spike|& number of grid per X-grid segment for the sharp peaks.\\
\verb|ny_init_guess|& number of grid per Y-grid segment in the initiative guess algorithm\\
\verb|ny_adapt|& number of grid per Y-grid segment in the adaptive algorithm\\
\verb|x_fine|& \textcolor{red}{optional}. Manually refine the X-grid. You can specify several sections by \verb|{}| wrapper. For example \verb|x_fine = {2.19:2.21%1000, 2.29:2.31%1000}|, where \verb|%1000| means you want to use 1000 times finer grid for this section.\\
\verb|y_fine|& \textcolor{red}{optional}. Manually refine the Y-grid. Usage the same as \verb|x_fine|.\\
\verb|eqnsolv_method|& How to solve the MQDT equation. Options are: hyb, bisect, newton.\\
\verb|relax|& \textcolor{red}{optional}. Relaxation coefficient to prevent oscillation. Default 1.0. \\
\verb|dmu|& \textcolor{red}{optional}. Constant correction to $\mu_{\alpha}$. Specify it with \verb|{}| wrapper: \verb|dmu = {0.0 0.0 0.0 0.0 0.0 0.0}| \\
\verb|au_peak_search_method|& \textcolor{red}{For postmqdt}. Either search the steepest change of each collisional-channel phase shift or the total   phase shift. Options: tot / chan. \\
\verb|k_mat_cofact|& the column index for cofactor when solving the A-coefficients from the secular equations (asymptotic boundary condition).
$\forall i \leq nchan$ will work, this is only left for debugging purpose. \\
\verb|twoJ|& \textcolor{red}{Not used currently}. $2j$ of the system, used to calculate Lande-g factor.\\
\hline
\end{tabular}
\end{table}
\newpage
Below is a sample input file.
\lstinputlisting[language=python]{mqdt.in}
The experimental resonances are prepared in the \emph{elev.exp}, the format is shown as below
\begin{lstlisting}[language=bash]
Unit of energy levels
Index Energy_value Eigenchann_index
\end{lstlisting}
Where index is the order of the \textcolor{red}{appearance} of the level in \textcolor{red}{eigenchannel}. While the \textcolor{red}{order of these level lines is free}. One example: 
\lstinputlisting[language=python]{elev.exp}

The \emph{command line} to employ the code:
\begin{lstlisting}[language=bash]
# to list helper information
postmqdt.exe --help 
# to specify working directory (where MQDT results are)
postmqdt.exe -p GIVE_IT_A_DIRECTORY
# otherwise it will look for input file in Present_Working_Directory by default
postmqdt.exe
# to specify debug output
# this will print all debug information
postmqdt.exe -debug all
# this will print interpolation and level(resonance) searching procedure
postmqdt.exe -debug interp+levfind
# run postmqdt.exe --help will list all debug options
\end{lstlisting}

\section{\label{sec:output}Program output}
The output file \emph{debug-post.out} contains the calculation parameter summation 
\lstinputlisting[language=python]{debug-post.out}

The file \emph{tau2.out} contains the calculated effective quantum defects/phase shifts, \emph{theo-plot.out} contains the theoretical levels(resonances).
One can run the following gnuplot script to plot it:
\lstinputlisting[language=python]{plot6ch-draw.gnu}
\begin{figure}[b]
	\includegraphics[width=\figurewidth]{{Oxygen-au-post}.png} 
	\caption{\label{fig:example}Solution of Oxygen $J^{\pi} = 1^-$ with 6 physical channels in autoionization region.}
\end{figure}

Other important output files are: 
\begin{lstlisting}[language=bash]
dbg_sum_dtaudE.out    # autoionization resonance phase shift slope
dfde2-plot.out        # oscillator strength density rearranged by eigenchannel 
os.out                # oscillator strength density
phystable.out         # Physcal observables: discrete levels/resonances
tau2-plot.out         # phase shifts rearranged by eigenchannel
debug-post.out        # calculation parameter summation
sig.out               # cross section
theo-plot.out         # theretical levels/resonances for plotting.
[exp-plot.out]        # experiment levels/resonances for plotting. (If elev.exp is provided.)
[theo-exp-compare.out]# comparison of theoretical and experimental data.(If elev.exp is provided.)
[nux_nuy.out]         # If in discrete region
\end{lstlisting}

Similarly, the results for discrete energy region is shown in Fig.~\ref{fig:dis}:
\begin{figure}[b]
	\includegraphics[width=\figurewidth]{{jhangz-dis-post}.png} 
	\caption{\label{fig:dis}Solution of Oxygen $J^{\pi} = 1^-$ with 6 physical channels in discrete region.}
\end{figure} 

\normalem
%\nocite{*}
\bibliography{mqdt}
\end{document}
%
% ****** End of file apssamp.tex ******

